\documentclass[a4paper]{report}
\usepackage[utf8]{inputenc}
\usepackage[portuguese]{babel}
\usepackage{hyperref}
\usepackage{a4wide}
\hypersetup{pdftitle={AP - Rooms},
pdfauthor={João Teixeira, José Ferreira},
colorlinks=true,
urlcolor=blue,
linkcolor=black}
\usepackage{subcaption}
\usepackage{listings}
\usepackage{booktabs}
\usepackage{multirow}
\usepackage{appendix}
\usepackage{tikz}
\usepackage{authblk}
\usepackage{bashful}
\usepackage{verbatim}
\usepackage{amssymb}
\usepackage{multirow}
\usepackage{mwe}
\usepackage[parfill]{parskip}
\usetikzlibrary{positioning,automata,decorations.markings}
\AfterEndEnvironment{figure}{\noindent\ignorespaces}
\AfterEndEnvironment{table}{\noindent\ignorespaces}

\begin{document}

\title{Algoritmos Paralelos\\Room Assignment Problem}
\author{João Teixeira (A85504) \and José Filipe Ferreira (A83683)}
\date{\today}

\begin{center}
    \begin{minipage}{0.75\linewidth}
        \centering
        \includegraphics[width=0.4\textwidth]{images/eng.jpeg}\par\vspace{1cm}
        \vspace{1.5cm}
        \href{https://www.uminho.pt/PT}
        {\color{black}{\scshape\LARGE Universidade do Minho}} \par
        \vspace{1cm}
        \href{https://www.di.uminho.pt/}
        {\color{black}{\scshape\Large Departamento de Informática}} \par
        \vspace{1.5cm}
        \maketitle
    \end{minipage}
\end{center}

\tableofcontents

\pagebreak

\chapter{Introdução}
O algoritmo escolhido para o projeto da unidade curricular de Computação
Paralela e Distribuída foi o \textit{Bucket Sort}.

Numa primeira fase do trabalho desenvolvemos uma versão sequencial do projeto e
procedemos ao \textit{benchmarking} do programa resultante. Em seguida
convertemos a implementação sequencial numa versão com utilização de memoria
partilhada fazendo uso de \textit{OpenMP} e comparamos o resultado com a versão
sequencial desenvolvida anteriormente.

Nesta segunda fase desenvolvemos uma nova versao do \textit{bucket sort} fazendo
uso de memoria distribuída com o \textit{OpenMPI} comparando os resultados do
benchmarking deste algoritmo com os resultados obtidos na fase anterior.

De notar que todos os benchmarks descritos foram efeutados em nós do tipo 642 do
cluster \textit{SeARCH}, e todos os executáveis foram compilados o
\textit{MPIcc} na versão 1.8.1 do \textit{OpenMPI}, utilizando a versão
7.2.0 do \textit{gcc}, e com as flags \textit{-O3 -std=c11} para garantir
a maior consistência entre os diferentes benchmarks. Na
ausência de indicação, os benchmarks foram efetuados com um input de 100000000
de elementos aleatórios, entre -1000 e 500000.

\chapter{OpenMPI} \label{chap:ompi}

\section{Descrição da Implementação}

\section{Análise dos resultados}
\begin{table}[h]
    \centering
    \begin{tabular}{|c|c|}
        \hline
        Processes & Time(Seconds) \\ \hline
        2         & 10,170775     \\ \hline
        4         & 10,061176     \\ \hline
        8         & 6,581753      \\ \hline
        16        & 2,5495        \\ \hline
        32        & 6,165742      \\ \hline
    \end{tabular}
    \caption{\label{tab:Times}Tempos de execução do algoritmo, com 10 baldes}
\end{table}

\appendix

\chapter{Algoritmo em dois nodos distintos}


\chapter{OpenMP vs OpenMPI}

\end{document}
