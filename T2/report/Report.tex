\documentclass[a4paper]{report}
\usepackage[utf8]{inputenc}
\usepackage[portuguese]{babel}
\usepackage{hyperref}
\usepackage{a4wide}
\hypersetup{pdftitle={AP - Poisson},
pdfauthor={João Teixeira, José Ferreira},
colorlinks=true,
urlcolor=blue,
linkcolor=black}
\usepackage{subcaption}
\usepackage{listings}
\usepackage{booktabs}
\usepackage{multirow}
\usepackage{appendix}
\usepackage{tikz}
\usepackage{authblk}
\usepackage{bashful}
\usepackage{verbatim}
\usepackage{amssymb}
\usepackage{multirow}
\usepackage{mwe}
\usepackage[parfill]{parskip}
\usetikzlibrary{positioning,automata,decorations.markings}
\AfterEndEnvironment{figure}{\noindent\ignorespaces}
\AfterEndEnvironment{table}{\noindent\ignorespaces}

\begin{document}

\title{Algoritmos Paralelos\\Numerical Solution of the Poisson Equation}
\author{João Teixeira (A85504) \and José Filipe Ferreira (A83683)}
\date{\today}

\begin{center}
    \begin{minipage}{0.75\linewidth}
        \centering
        \includegraphics[width=0.4\textwidth]{images/eng.jpeg}\par\vspace{1cm}
        \vspace{1.5cm}
        \href{https://www.uminho.pt/PT}
        {\color{black}{\scshape\LARGE Universidade do Minho}} \par
        \vspace{1cm}
        \href{https://www.di.uminho.pt/}
        {\color{black}{\scshape\Large Departamento de Informática}} \par
        \vspace{1.5cm}
        \maketitle
    \end{minipage}
\end{center}

\tableofcontents

\pagebreak

\chapter{Introdução}
A equação de Poisson é usada nas áreas das ciências naturais e da física
teórica. A solução desta equação pode ser usada para, por exemplo, simular os
campos electromagnéticos gerados por partículas electricamente carregadas. Esta
é uma generalização da equação de Laplace que também tem múltiplos usos na
física teórica.

A formulação desta equação a duas dimensões e dada por:

\[ \left( \frac{\partial^2 u}{\partial x^2} + \frac{\partial^2 u}{\partial y^2} \right)= g(x, y) \]

Para simplificar assume-se que g(x,y) = 0 sendo que u representa o limite da
área. Os valores da solução não são computados de forma continua mas sim de
forma discreta.

Ao longo deste relatório iremos descrever os vários algoritmos iterativos
desenvolvidos em C com o objetivo de aproximar a solução desta equação assim
como a estratégia de paralelização utilizada com vista a melhorar a performance
dos mesmos culminando numa analise de tempos de execução.

\chapter{Algoritmos Desenvolvidos}

Foram desenvolvidos três algoritmos iterativos distintos para resolver este
problema. Gauss-Seidel, Gauss-Seidel com Red-Black e Successive Overrelaxation
com Red-Black.

Todos os algoritmos recebem como parâmetro o tamanho do vetor a resolver e o
valor de tolerância e devolvem o vetor com os valores aproximados e o numero de
iterações que tiveram de ocorrer para calcular esse valor.

O vetor que contem o que foi calculado até ao momento denomina-se de w e o vetor
que contem o que foi calculado na iteração anterior denomina-se de u. O vetor w
é populado no inicio com todos os valores na borda com o valor 100, menos na
borda superior em que os valores são 0, e com os valores interiores iguais a 50.

Numa primeira tentativa de melhorar a performance do algoritmo, paralelizou-se o
ato de popular a matriz. Visto que o algoritmo inicial já vetorizava

Para todos os algoritmos o caso de paragem consiste em subtrair o vetor w com o
u e procurar qual é o valor maior em modulo. Esse modulo é comparado com o valor
de tolerância passado por argumento. Caso o valor de tolerância seja maior do
que o valor calculado então o algoritmo iterativo  acaba.

Desta forma, a única diferença entre estes algoritmos na sua forma sequencial
encontra-se na forma como eles percorrem e calculam o vetor w em cada iteração.

\section{Gauss-Seidel}

O algoritmo de Gauss-Seidel consiste em percorrer todos os pontos do vetor w
linha por linha. Para cada ponto soma-se o ponto que está diretamente acima, o
ponto abaixo, o ponto à esquerda e o ponto à direita e divide-se o resultado por
4. Desta forma o valor do ponto é substituído pela média dos pontos diretamente
adjacentes.

Para parelelizar este algoritmo, notamos que era possivel calcular os pontos de
uma diagonal em simultaneo. Entao criamos um ciclo que itera por cada diagonal,
e calculamos cada ponto desta em paralelo.

\begin{figure}[h]
    \centering
        \includegraphics[width=0.4\textwidth]{images/wave.jpeg}
        \caption{Representação visual do algoritmo}
        \label{fig:wave}
\end{figure}

\pagebreak

\section{Gauss-Seidel com Red-Black}

Tal como se viu o algoritmo de Gauss-Seidel simples não é fácil de
paralelizar. Por isso, foi criado uma versão modificada do algoritmo que
percorre o vetor w de forma diferente para tentar mitigar este problema.

O método utilizado de percorrer o vetor é conhecido como Red-Black porque o
resultado se assemelha a um tabuleiro de xadrez com as casas de cor vermelha e
preta. Dentro de cada iteração primeiro percorre-se posição sim posição não de
cada linha par e posição não posição sim de cada linha par (células vermelhas da
imagem \ref{fig:chess}) e depois percorre-se posição não posição sim de cada
linha par e posição sim posição não de cada linha ímpar (células pretas da
imagem \ref{fig:chess}). Em cada iteração, para cada uma das posições do vetor w
o novo valor é dado na mesma por calcular a média dos pontos diretamente
adjacentes

\begin{figure}[h]
    \centering
        \includegraphics[width=0.4\textwidth]{images/chess.jpg}
        \caption{tabuleiro de xadrez}
        \label{fig:chess}
\end{figure}

Desta forma cada uma destas duas passagens dentro de cada iteração não tem
dependências de dados dentro de si mesmas.

Visto que não existem dependências de dados entre calcular as celulas vermelhas
e as celulas pretas, é possivel dentro de cada iteração calcular todos os pontos
de uma cor em simultaneo. Tomando partido disto, aplicamos a paralelização nos
ciclos responsaveis por calcular cada uma das cores.

\section{Successive Overrelaxation com Red-Black}

A versão sequencial deste algoritmo é muito próxima da versão sequencial do
algoritmo Gauss-Seidel com Red-Black. No Successive Overrelaxation com Red-Black
o que difere é a parte como é calculado o novo valor do vetor w. Neste caso,
para além de se ter em conta a média dos quatro pontos adjacentes ao ponto que
se está a calcular, também se tem em conta o próprio valor seguindo a formula:

\[ p = \frac{2}{1 + sin{\frac{\pi}{SIZE - 1}}} \]
\[novo\_valor = (1-p) * pixel + p * media \]

Esta mudança implica que o algoritmo tende mais rapidamente para o caso de
paragem e, por isso, tende a ter menos iterações.

Como a iteração sobre o vetor w é semelhante à usada em Gauss-Seidel com
Red-Black, a paralelização deste algoritmo foi efetuada da mesma forma, descrita
acima.

\chapter{Representação de Resultados}

Para representar os resultados obtidos pelos algoritmos criados, foi feita uma
função que representa o vetor w sobre a forma de uma imagem no formato
\textit{portable pixmap format} (PPM). Este formato foi escolhido devido ao
facto de suportar formato ASCII tornando a escrita da imagem substancialmente
mais simples. O formato indica que existem duas áreas no ficheiro. Uma primeira
área que indica o formato utilizado e uma segunda área com os pontos
propriamente ditos.

A área do formato contem três linhas. A primeira linha representa o formato em
que a imagem foi escrita. Neste caso o formato escolhido foi o formato P3 que
indica que a imagem está escrita em ASCII e que representa pontos RGB. A segunda
linha indica as dimensões da imagem separada por espaço. Por fim, a ultima linha
representa o valor máximo de cada valor de cor. Neste caso o valor escolhido
foi 255.

A área com os pontos propriamente ditos contem um ponto por linha. Sendo que
cada linha contem o valor R, o valor G e o valor B do pixel separado por espaços.

Desta forma, para representar uma imagem com apenas um pixel branco basta fazer:
\begin{verbatim}
P3
1 1
255
255 255 255
\end{verbatim}

Para representar os pontos do vetor estes são primeiro convertidos para HSV.
Neste passo a Saturação e Valor são sempre mantidos constantes para todos os
pontos, sendo que o único valor que varia é o Hue. Para existir uma variação
constante de azul para amarelo com o aumento dos valores é necessário que o Hue
varie entre 60 e 240. O próximo passo consiste em converter os valores dentro do
vetor w para estarem contidos dentro desse intervalo. Por fim pode-se converter
o HSV obtido para RGB e colocar no ficheiro.

\begin{figure}[h]
\centering
\begin{minipage}{.3\textwidth}
  \centering
  \includegraphics[width=.95\linewidth]{images/poisson_gs_100.png}
\end{minipage}%
\begin{minipage}{.3\textwidth}
  \centering
  \includegraphics[width=.95\linewidth]{images/poisson_gs_500.png}
\end{minipage}
\begin{minipage}{.3\textwidth}
  \centering
  \includegraphics[width=.95\linewidth]{images/poisson_gs_1000.png}
\end{minipage}
    \caption{Comparação de diferentes outputs com diferentes tamanhos e
    tolerância constante}
\end{figure}

\chapter{Analise de Performance}

Os testes foram corridos no cluster Search da universidade do Minho nas máquinas
da fila cpar, nomeadamente as máquinas 652. Os valores foram obtidos fazendo uso
do k-best com k igual a 8. O programa foi compilado com -O3.


\begin{table}[h]
\centering
\begin{tabular}{|l|l|l|}
\hline
                                                 & num iterações & tempo (segundos) \\ \hline
Gauss-Seidel                                     & 922           & 0.050               \\ \hline
Gauss-Seidel paralelo                            & 922           & 135.080               \\ \hline
Gauss-Seidel com Red-Black                       & 540           & 0.020               \\ \hline
Gauss-Seidel com Red-Black paralelo              & 915           & 1.720               \\ \hline
Successive Overrelaxation com Red-Black          & 571           & 0.010               \\ \hline
Successive Overrelaxation com Red-Black paralelo & 571           & 0.280               \\ \hline
\end{tabular}
\caption{Comparação de Performance (para N=100 e TOL=0.001)}
\label{tab:tempo}
\end{table}

\begin{table}[h]
\centering
\begin{tabular}{|l|l|l|}
\hline
                                                 & num iterações & tempo (segundos) \\ \hline
Gauss-Seidel                                     & 923           & 1.670            \\ \hline
Gauss-Seidel paralelo                            & 923           & 516.480          \\ \hline
Gauss-Seidel com Red-Black                       & 610           & 0.440               \\ \hline
Gauss-Seidel com Red-Black paralelo              & 902           & 23.270               \\ \hline
Successive Overrelaxation com Red-Black          & 126           & 0.460            \\ \hline
Successive Overrelaxation com Red-Black paralelo & 126           & 14.660           \\ \hline
\end{tabular}
\caption{Comparação de Performance (para N=500 e TOL=0.001)}
\label{tab:tempo}
\end{table}

\begin{table}[h]
\centering
\begin{tabular}{|l|l|l|}
\hline
                                                 & num iterações & tempo (segundos) \\ \hline
Gauss-Seidel                                     & 922           & 7.030            \\ \hline
Gauss-Seidel paralelo                            & 922           & 1189.080         \\ \hline
Gauss-Seidel com Red-Black                       & 610           & 2.140                 \\ \hline
Gauss-Seidel com Red-Black paralelo              & 902           & 153.540                 \\ \hline
Successive Overrelaxation com Red-Black          & 1107          & 3.870            \\ \hline
Successive Overrelaxation com Red-Black paralelo & 1107          & 123.730          \\ \hline
\end{tabular}
\caption{Comparação de Performance (para N=1000 e TOL=0.001)}
\label{tab:tempo}
\end{table}

\end{document}
